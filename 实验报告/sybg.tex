\documentclass[UTF8,a4paper]{ctexart}
\usepackage{ulem}
\usepackage[left=1.25in,right=1.25in,top=1in,bottom=1in]{geometry}
\usepackage{listings}
\usepackage{enumitem}
\newfontfamily\courier{Courier New}
\newfontfamily\courierb{Courier New Bold}
\lstset{
language=C++,
columns=fullflexiblem,
breaklines=true,
%numbers=left,%在左侧显示行号
frame=none,%不显示背景边框
backgroundcolor=\color[RGB]{245,245,244},%设定背景颜色
keywordstyle=\courierb,%设定关键字颜色
%numberstyle=\consolas\color{darkgray},%设定行号格式
%commentstyle=\color[RGB]{0,96,96},%设置代码注释的格式
%stringstyle=\color{myorange},%设置字符串格式
showstringspaces=false,%不显示字符串中的空格
basicstyle=\courier,%\color{mygreen},
extendedchars=false 
}
\linespread{1}
\begin{document}
\pagestyle{plain}
  \centerline{\bfseries\zihao{3}南京信息工程大学\ 操作系统\ 实验报告}
  \zihao{5}
  \begin{center}
    实验名称 \uline{\enskip 操作系统 \enskip }日期 \uline{\enskip 2015/5/10\enskip}得分 \uline{\qquad\qquad}  指导教师 \uline{\enskip 毕硕本 \enskip}
  \end{center}
  \begin{center}
    系 \uline{\enskip 理学系 \enskip} 专业 \uline{\enskip 信息与计算科学 \enskip} 班级 \uline{\enskip 1 \enskip}
    姓名 \uline{\enskip 王星晨 \enskip } 学号 \uline{\enskip 20152314026 \enskip}
  \end{center}
  \zihao{-4}
  {\noindent\zihao{4}\bfseries{1.实验目的}}
\begin{itemize}
\item 理解进程同步的信号量机制
\item 学会用C/C++实现进程同步的信号量机制
\item 学会利用信号量机制解决经典的进程同步问题
\item 学会调用操作系统API函数创建进程
\end{itemize}

 {\noindent\zihao{4}\bfseries{2.实验内容}}
\begin{enumerate}[label=(\arabic*)]
  \item 利用记录型信号量解决生产者 - 消费者问题
  \item 利用AND信号量机制解决哲学家进餐问题
  \item 利用记录型信号量解决读者写着问题
\end{enumerate}

{\noindent\zihao{4}\bfseries{3.实验思路}}

1.生产者消费者问题
假设生产者和消费者之间的公用缓冲池具有n个缓冲区,可以利用互斥信号量mutex实现诸进程对缓冲池的使用;利用信号量empty和full 分别表示缓冲池中空缓冲区和满缓冲区数量。又假定这些生产者和消费者相互等效,只要缓冲池未满,生产者便可将消息送入缓冲池;只要缓冲池位空,消费者便可从缓冲池中取走一个消息。
定义信号量类型semaphore为整型。为了方便,假设生产的产品为一个随机整数,则item 类型定义为整型。程序的变量和类型定义为
\begin{lstlisting}
const int n=5;//缓冲区数量规定为5
int in=0,out=0;
typedef int item;
typedef int semaphore;
semaphore mutex=1,Empty=n,full=0;
item buffer[n];
\end{lstlisting}
对于P,V操作,定义为
\begin{lstlisting}
void wait (semaphore &S)
{
	while(S<=0);
	S--;
}
void signal (semaphore &S)
{
	S++;
}
\end{lstlisting}
producer和consumer是两个同时进行的进程,需要调用操作系统API来实现.
为了方便,调用CreateThread函数创建producer和consumer线程来模拟进程.
\begin{lstlisting}
HANDLE handle[2];
handle[0]=CreateThread(NULL,0,producer,NULL,0,NULL);
handle[1]=CreateThread(NULL,0,concumer,NULL,0,NULL);
WaitForMultipleObjects(2,handle,TRUE,-1);
\end{lstlisting}

2.哲学家进餐问题

在哲学家进餐问题中,要求每个哲学家先获得两个临界资源(筷子)后方能进餐,这在本质上是AND同步问题。故用AND信号量机制可获得最简洁的解法。
Swait和Ssignal定义如下
\begin{lstlisting}
  int Swait(semaphore &S1, semaphore &S2)
{
	while (true)
	{
		if (S1 >= 1 && S2 >= 1)
		{
			S1--;
			S2--;
			break;

		}
		else
		{
			;
		}

	}
	return 0;
}
int Ssignal(semaphore &S1, semaphore &S2)
{
	S1++;
	S2++;

	return 0;
}
\end{lstlisting}

3.读者 - 写者问题

为实现Reader与Writer进程间在读或写的互斥而设置一个互斥信号量wmutex.另外,再设置一个整型变量readcount表示在读的进程数目。由于只要有一个Reader进程在读,便不允许Writer进程去写。因此,仅当readcount=0,表示尚无Reader进程在读时,Reader 进程才需执行wait(wmutex)操作。若wait(wmutex)操作成功,Reader进程便可去读,相应的,做readcount+1操作。同理,仅当Reader进程在执行了readcount减1操作后其值为0时,才需执行signal(wmutex)操作,以便让Writer进程写操作。又因为readcount是一个可被多个Reader进程访问的资源,因此,也应该为它设置一个互斥信号量mutex.

  {\noindent\zihao{4}\bfseries{4.实验步骤}}
  
\noindent 1.生产者消费者问题
\begin{enumerate}[label=(\arabic*)]
\item 打开Visual Studio ,建立C++项目,添加ProducerAndConsumer.cpp文件
\item 编写生产者消费者程序代码
\item 调试运行
\end{enumerate}
\noindent 2.哲学家进餐问题
\begin{enumerate}[label=(\arabic*)]
\item 打开Visual Studio ,建立C++项目,添加Philosopher.cpp文件
\item 编写哲学家进餐问题程序代码
\item 调试运行
\end{enumerate}
\noindent 3.读者写着问题
\begin{enumerate}[label=(\arabic*)]
\item 打开Visual Studio ,建立C++项目,添加ReadWrite.cpp文件
\item 编写读者写者问题程序代码
\item 调试运行
\end{enumerate}
    {\noindent\zihao{4}\bfseries{5.调试结果}}

\noindent 1.消费者生产者问题

\centerline{程序在屏幕的部分输出结果}
\begin{lstlisting}[frame=shadowbox,backgroundcolor=\color{white}]
producer an item: 41 in buffer 0
concumer an item: 41 in buffer 0
producer an item: 18467 in buffer 1
concumer an item: 18467 in buffer 1
producer an item: 6334 in buffer 2
concumer an item: 6334 in buffer 2
producer an item: 26500 in buffer 3
concumer an item: 26500 in buffer 3
producer an item: 19169 in buffer 4
concumer an item: 19169 in buffer 4
producer an item: 15724 in buffer 0
concumer an item: 15724 in buffer 0
producer an item: 11478 in buffer 1
concumer an item: 11478 in buffer 1
producer an item: 29358 in buffer 2
concumer an item: 29358 in buffer 2
producer an item: 26962 in buffer 3
concumer an item: 26962 in buffer 3
\end{lstlisting}

producer不断产生产品(整数)放入缓冲区buffer[i],consumer不断在缓冲区buffer消费产品

\noindent 2.哲学家进餐问题

\centerline{程序在屏幕的部分输出结果}
\begin{lstlisting}[frame=shadowbox,backgroundcolor=\color{white}]
philosopher 1 start eating
philosopher 3 start eating
philosopher 1 start thinking
philosopher 5 start eating
philosopher 3 start thinking
philosopher 2 start eating
philosopher 5 start thinking
philosopher 4 start eating
philosopher 2 start thinking
philosopher 1 start eating
philosopher 4 start thinking
philosopher 3 start eating
philosopher 1 start thinking
philosopher 5 start eating
philosopher 3 start thinking
\end{lstlisting}

相邻的哲学家是不能同时进餐的,5个哲学家最多只允许2个哲学家同时进餐,程序结果符合这个特点。

\noindent 3.读者写者问题

\centerline{程序在屏幕的部分输出结果}
\begin{lstlisting}[frame=shadowbox,backgroundcolor=\color{white}]
Reading...
Finish reading.
Writing...
Finish writing.
Reading...
Finish reading.
Writing...
Finish writing.
Reading...
Finish reading.
Writing...
Finish writing.
Reading...
Finish reading.
Writing...
Finish writing.
Reading...
\end{lstlisting}

读和写不能同时进行,程序结果符合这个特点.


       {\noindent\zihao{4}\bfseries{6.代码附录}}
       
\noindent 1.生产者消费者问题 

\centerline{ProducerAndConsumer.cpp}
\lstinputlisting[language=C++]{../ProducerAndConsumer/ProducerAndConsumer.cpp}

\noindent 2.哲学家进餐问题

\centerline{Philosopher.cpp}
\lstinputlisting[language=C++]{../Philosopher/main.cpp}

\noindent 3.读者写者问题

\centerline{ReadWrite.cpp}
\lstinputlisting[language=C++]{../ReadWrite/ReadWrite/main.cpp}
\end{document}
